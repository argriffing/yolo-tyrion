\documentclass{article}

\usepackage[authoryear,round]{natbib}
\usepackage{amsmath}
\usepackage{amssymb}
\usepackage{amsthm}

\usepackage{float}
\usepackage{graphicx}
\usepackage{caption}
\usepackage{subcaption}
\usepackage{tikz}
\usepackage{hyperref}

\theoremstyle{plain}
\newtheorem{thm}{Theorem}[section]
\newtheorem{lem}[thm]{Lemma}
\newtheorem{trick}[thm]{Trick}

\theoremstyle{definition}
\newtheorem{defn}[thm]{Definition}
\newtheorem{prop}[thm]{Property}

\usetikzlibrary{decorations.pathreplacing}

% the default max matrix dimensions in amsmath is ten
\setcounter{MaxMatrixCols}{20}

% this is copied from the internet
% and its purpose is to compensate for bad vertical spacing in amsmath pmatrix
% by letting you do things like \begin{pmatrix}[1.5]
\makeatletter
\renewcommand*\env@matrix[1][\arraystretch]{%
	  \edef\arraystretch{#1}%
	    \hskip -\arraycolsep
	      \let\@ifnextchar\new@ifnextchar
	        \array{*\c@MaxMatrixCols c}}
		\makeatother


\begin{document}

\title
{
	Decompositions of rate matrices of continuous-time Moran
	models of allele frequency change in tiny populations,
	for specific weird mutational structures
}
% \author{Alexander Griffing}
\maketitle

\section{Jordan decompositions for tiny populations of size $N$}

In each of these examples,
the mutation rate is scaled so that the equilibrium distribution
of the compound state \{AB + Ab\} is discrete uniform.

\subsection{$N=1$}

When the population size is $1$, only the mutation process matters,
and genetic drift does not affect the rate matrix.
The population genetic rate matrix is just the mutation rate matrix.

Population genetic rate matrix for $N = 1$:
\begin{equation}
	Q =
	\begin{pmatrix}
		-2 & 1 & 1 & 0 \\
		1 & -2 & 0 & 1 \\
		1 & 0 & -2 & 1 \\
		0 & 1 & 1 & -2
	\end{pmatrix}
\end{equation}

Jordan decomposition:
\begin{equation}
	Q =
	\begin{pmatrix}
		1 & -1 & 0 & 1 \\
		-1 & 0 & -1 & 1 \\
		-1 & 0 & 1 & 1 \\
		1 & 1 & 0 & 1
	\end{pmatrix} \\
	\begin{pmatrix}
		-4 & 0 & 0 & 0 \\
		0 & -2 & 0 & 0 \\
		0 & 0 & -2 & 0 \\
		0 & 0 & 0 & 0
	\end{pmatrix}
	\begin{pmatrix}
		1/4 & -1/4 & -1/4 & 1/4 \\
		-1/2 & 0 & 0 & 1/2 \\
		0 & -1/2 & 1/2 & 0 \\
		1/4 & 1/4 & 1/4 & 1/4
	\end{pmatrix}
\end{equation}


\subsection{$N = 2$}

Population genetic rate matrix for $N = 1$:
\begin{equation}
	Q =
	\begin{pmatrix}
		-4 & 2 & 0 & 2 & 0 & 0 & 0 & 0 & 0 & 0 \\
		2 & -6 & 2 & 0 & 1 & 0 & 1 & 0 & 0 & 0 \\
		0 & 2 & -4 & 0 & 0 & 0 & 0 & 2 & 0 & 0 \\
		2 & 0 & 0 & -6 & 1 & 2 & 1 & 0 & 0 & 0 \\
		0 & 1 & 1 & 1 & -6 & 1 & 0 & 1 & 1 & 0 \\
		0 & 0 & 0 & 2 & 0 & -4 & 0 & 0 & 2 & 0 \\
		1 & 1 & 0 & 1 & 0 & 0 & -6 & 1 & 1 & 1 \\
		0 & 0 & 2 & 0 & 1 & 0 & 1 & -6 & 0 & 2 \\
		0 & 0 & 0 & 0 & 1 & 2 & 1 & 0 & -6 & 2 \\
		0 & 0 & 0 & 0 & 0 & 0 & 0 & 2 & 2 & -4
	\end{pmatrix}
\end{equation}

Jordan decomposition $Q = P J P^{-1}$:
\begin{align}
	P &=
	\begin{pmatrix}
		2 & -1 & 0 & 1 & 0 & 0 & 1 & -1 & 0 & 1 \\
		-3 & 0 & -1 & 0 & 1 & 0 & 0 & 0 & -1 & 1 \\
		2 & 1 & 1 & 1 & 0 & 0 & -1 & 1 & -2 & 1 \\
		-3 & 2 & 1 & -1 & -1 & 0 & 0 & -1 & 1 & 1 \\
		2 & 0 & 0 & -4 & 0 & -1 & -1 & 0 & 0 & 1 \\
		2 & -1 & -1 & 1 & 0 & 0 & -1 & -1 & 2 & 1 \\
		2 & 0 & 0 & 0 & 0 & 1 & 1 & 0 & 0 & 1 \\
		-3 & -2 & -1 & -1 & -1 & 0 & 0 & 1 & -1 & 1 \\
		-3 & 0 & 1 & 0 & 1 & 0 & 0 & 0 & 1 & 1 \\
		2 & 1 & 0 & 1 & 0 & 0 & 1 & 1 & 0 & 1
	\end{pmatrix} \\
	J &=
	\begin{pmatrix}
		-10 & 0 & 0 & 0 & 0 & 0 & 0 & 0 & 0 & 0 \\
		0 & -8 & 0 & 0 & 0 & 0 & 0 & 0 & 0 & 0 \\
		0 & 0 & -8 & 0 & 0 & 0 & 0 & 0 & 0 & 0 \\
		0 & 0 & 0 & -6 & 0 & 0 & 0 & 0 & 0 & 0 \\
		0 & 0 & 0 & 0 & -6 & 0 & 0 & 0 & 0 & 0 \\
		0 & 0 & 0 & 0 & 0 & -6 & 0 & 0 & 0 & 0 \\
		0 & 0 & 0 & 0 & 0 & 0 & -4 & 0 & 0 & 0 \\
		0 & 0 & 0 & 0 & 0 & 0 & 0 & -2 & 0 & 0 \\
		0 & 0 & 0 & 0 & 0 & 0 & 0 & 0 & -2 & 0 \\
		0 & 0 & 0 & 0 & 0 & 0 & 0 & 0 & 0 & 0
	\end{pmatrix} \\
	P^{-1} &=
	\begin{pmatrix}
		1/40 & -1/20 & 1/40 & -1/20 & 1/20 & 1/40 & 1/20 & -1/20 & -1/20 & 1/40 \\
		-1/6 & 1/6 & 0 & 1/6 & 0 & 0 & 0 & -1/6 & -1/6 & 1/6 \\
		1/6 & -1/3 & 1/6 & 0 & 0 & -1/6 & 0 & 0 & 1/3 & -1/6 \\
		1/12 & 0 & 1/12 & 0 & -1/6 & 1/12 & -1/6 & 0 & 0 & 1/12 \\
		-1/24 & 1/4 & -1/24 & -1/4 & 1/12 & -1/24 & 1/12 & -1/4 & 1/4 & -1/24 \\
		-5/12 & 0 & 1/12 & 0 & -1/6 & 1/12 & 5/6 & 0 & 0 & -5/12 \\
		1/4 & 0 & -1/4 & 0 & 0 & -1/4 & 0 & 0 & 0 & 1/4 \\
		-1/3 & -1/6 & 0 & -1/6 & 0 & 0 & 0 & 1/6 & 1/6 & 1/3 \\
		-1/6 & -1/6 & -1/6 & 0 & 0 & 1/6 & 0 & 0 & 1/6 & 1/6 \\
		7/60 & 1/10 & 7/60 & 1/10 & 1/15 & 7/60 & 1/15 & 1/10 & 1/10 & 7/60
	\end{pmatrix}
\end{align}

\end{document}

